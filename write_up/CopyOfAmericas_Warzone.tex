\documentclass{article} % For LaTeX2e
\usepackage{nips15submit_e,times}
\usepackage{hyperref}
\usepackage{url}
\usepackage{lineno}
\usepackage{graphicx}
\usepackage{amsmath}
\usepackage{multicol}
\usepackage[all]{hypcap} 
\usepackage{listings}
\usepackage{float}
\usepackage{amsfonts}
\usepackage{multicol}
\usepackage{breqn}
\usepackage{bm}
\setlength{\columnsep}{1cm}
%\linenumbers% Uncomment for line numbers



\title{America's Warzone: Predicting Gun Violence in Chicago}

\author{
Reuben K. McCreanor\thanks{Department of Statistical Science, Duke University} \\  
\texttt{reuben.mccreanor@duke.edu} \\
\And
Anna Yanchenko\footnotemark[1] \\
\texttt{anna.yanchenko@duke.edu} \\
\And 
Lei Qian\footnotemark[1] \\
\texttt{lei.qian@duke.edu} \\
\And
Megan Robertson\footnotemark[1] \\
\texttt{megan.robertson@duke.edu} \\ 
}

\hypersetup{
    colorlinks=true,
    linkcolor=blue,
    filecolor=magenta,      
    urlcolor=cyan,
    pdftitle={Sharelatex Example},
    bookmarks=true,
    pdfpagemode=FullScreen,
}


\newcommand{\fix}{\marginpar{FIX}}
\newcommand{\new}{\marginpar{NEW}}

\nipsfinalcopy 

\begin{document}

\maketitle

\noindent The crimes data comes from the Citizen Law Enforcement and Analysis and Reporting (CLEAR) system. The iformation is based on what is supplied by the individuals reporting the crimes and thus may not be accurate. In addition, there is a potential for human error when entering the crimes. The data provides block level information to protect the privacy and identity of those involved in crimes. \newline

\noindent Additional data sources were used for co-variates in the contruction of models. These included Affordable Rental Housing Developments, Census Data – Selected socioeconomic indicators, Average Electricity Usage, Average Gas Usage per Square Foot by Community Area, 311 Service Requests – Graffiti Removal, Public Health Statistics, and Vacant Properties by Community Area. \hyperref{}[] The data can be found on the same website as the Chicago Crime Data. Data from the 2010 census was also used for additional data. \hyperref{}[] The additional data is used to provide more information about the characteristics of Chicago neighborhoods. For example, socioeconomic indicators are related to crime rate since more crimes tend to occur in lower income neighborhoods. More vacant properties could indicate that the area is neglected or run down.

\end{document}
